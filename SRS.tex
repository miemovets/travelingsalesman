\documentclass[a4paper, 12pt] {article}

\usepackage{geometry}
\geometry{a4paper, left=20mm, right=10mm, top=20mm, bottom=20mm}
\usepackage[T2A]{fontenc}
\usepackage[utf8]{inputenc}
\usepackage{amsmath}
\usepackage{amssymb}
\usepackage[english, russian]{babel}
\usepackage{setspace}
\renewcommand{\labelenumii}{\arabic{enumii}}
\onehalfspacing
\usepackage{fancybox,fancyhdr}
\usepackage{comment}
\usepackage{graphicx}
\graphicspath{{pictures/}}
\DeclareGraphicsExtensions{.pdf,.png,.jpg,.gif}
\usepackage{verbatim}
\fancyhead[L]{НИС <<Введение в специальность>>}
\fancyhead[R]{\thepage}
\begin{document}


\begin{titlepage}
\begin{center}
{\textsc{\textbf{ПРАВИТЕЛЬСТВО РОССИЙСКОЙ ФЕДЕРАЦИИ}}}\\
\vspace{0.5cm}
\hrule
\vspace{0.5cm}
{\textsc{Федеральное государственное автономное образовательное учреждение высшего образования \\ <<Национальный исследовательский университет \\ <<Высшая школа экономики>>}}\\
\vspace{1cm}
Департамент прикладной математики
\end{center}

\vspace{\fill}

\begin{center}
{\Large{\textbf{ОТЧЕТ\\}}}
{\Large{\textbf{ПО ПРОДЕЛАННОЙ РАБОТЕ\\}}}
{\Large{\textbf{В РАМКАХ НАУЧНО-ИССЛЕДОВАТЕЛЬСКОГО СЕМИНАРА\\}}}
{\Large{\textbf{<<ВВЕДЕНИЕ В СПЕЦИАЛЬНОСТЬ>>\\}}}
\vspace{2em}
{\large{\textbf{ТЕМА:\\ ЗАДАЧА КОММИВОЯЖЁРА}}}
\end{center}

\vspace{\fill}


\begin{flushright}
Работу выполнил: Башкиров Данил Павлович\\
Научный руководитель: Посыпкин Михаил Анатольевич
\end{flushright}

\vspace{\fill}

\begin{center}
Москва~2018
\end{center}
\end{titlepage}
\pagestyle{fancy}
\section*{Введение}
Задача комивояжёра - задача, заключающаяся в поиске кратчайшего гамильтонова цикла. Гамильтонов цикл - цикл наименьшей длины или стоимости, проходящий через все вершины графа, при этом степень каждой вершины должна быть равна двум. \\

\noindent Точно неизвестно, когда задача была поставлена впервые, однако подобная проблема поднималась в книге 1832 года \textit{<<Коммивояжёр — как он должен вести себя и что должен делать для того, чтобы доставлять товар и иметь успех в своих делах — советы старого курьера>>}. В ней предлагались наилучшие маршруты в Германии и Швейцарии для странствующих торговцев. Во второй половине XX века задача коммивояжера привлекла внимание ученых из США и Европы. На ее основе были построены многие современные алгоритмы.

\newpage

\section{Методы решения}
\subsection{Муравьиный алгоритм}
Данный алгоритм моделирует поведение колонии муравьев при поиске пищи. В реальной жизни муравьи ходят случайными путями, а при обнаружении провизии прокладывают феромонные тропы. Они помогают другим муравьям находить дорогу к пище, что значительно ускоряет её добычу. Феромоны имеют свойство испаряться, но если путь пользуется популярностью, то она постоянно обновляется. Таким образом, если дорога занимает достаточно много времени, то она начинает слабеть. И наоборот, если феромонный уровень достаточно высок то это поможет найти кратчайший путь.
\begin{center}
\includegraphics{ants.png}
\end{center}
\newpage

\subsection{Генетический алгоритм}
Данный алгоритм моделирует механизм естесвенного отбора в природе. Генерируется случайный вектор генов, где каждый ген может быть представлен числом, битом и каким-то другим объектом. Случайным образом генерируется множество таких векторов и к нему применяется <<функция приспособленности>>, которая проверит приспособленность данного множества генотипов. В нашем случае это будет функция, которая будет проверять условие минимальной стоимости гамильтонова цикла. Каждому такому множеству присваивается оценка, насколько хорошо оно справляется с поставленной задачей. Выбираются несколько лучших решений и к ним снова применяется <<функция приспособленности>>. Этот набор действий подобен <<эволюционному процессу>>, и для его прекращения нужен критерий остановки. 
\begin{center}
\includegraphics[height=0.7\textheight]{genetic.png}
\end{center}

\subsection{Метод эластичной сети}
Данный алгоритм был предложен в 1987 году Дурбином и Уиллшоу, которые указали на его аналогичность механизмам установления упорядоченных нейронных связей. Суть заключается в установке небольшой окружности на плоскость, которая будет неравномерно расширяться, становясь, в итоге, искомым гамильтоновым циклом. У каждой точки окружности есть две цели: достичь ближайшего города и держаться как можно ближе к соседям, чтобы длина кольца была наименьшей. В дальнейшем города будут притягивать к себе только ближайшие точки окружности, так что конечная окружность будет искомым маршрутом.

\subsection{Метод ветвей и границ}
Данный метод впервые был предложен для решения задач целочисленного программирования в 1960 году Лендом и Дойгом. Алгоритм может быть рассмотрен на примере поиска минимума для функции $f(x)$. Разобьем множество определения $x$ на подмножества, и представим их в виде узлов дерева. Полученное дерево называется \textit{деревом поиска}. Тогда, если A и B - подмножества множества значений х, а $inf(f(A)) > sup(f(B))$, то множество А дальше можно не рассматривать. Если рекурсивно применить метод ко всем узлам дерева, то в итоге останется область, максимальное и нижнее  значение функции от которой будут совпадать. Именно эта область и будет минимумом.

\section{Ход решения}

\begin{enumerate}
\item Пусть дана матрица смежности связного графа:

\begin{tabular}{|c|c|c|c|c|c|}
\hline
& 1 & 2 & 3 & 4 & 5 \\
\hline
 1 & $\infty$ & 20 & 18 & 12 & 8 \\
\hline
2 & 5 & $\infty$ & 14 & 7 & 11 \\
\hline
3 & 12 & 18 & $\infty$ & 6 & 11 \\
\hline
4 & 11 & 17 & 11 & $\infty$ & 12 \\
\hline
5 & 5 & 5 & 5 & 5 & $\infty$ \\
\hline
\end{tabular}

Для того, чтобы исключить зацикливание, элементам, лежащим на главной диагонали присваивается бесконечно большое значение.


\item Первым этапом алгоритма является редуцирование матрицы по строкам и столбцам. Для этого для каждой строки найдем минимальный элемент $d_{i}$:

\begin{tabular}{|c|c|c|c|c|c|c|}
\hline
& 1 & 2 & 3 & 4 & 5 & $d_{i}$\\
\hline
 1 & $\infty$ & 20 & 18 & 12 & 8 & 8\\
\hline
2 & 5 & $\infty$ & 14 & 7 & 11 & 5\\
\hline
3 & 12 & 18 & $\infty$ & 6 & 11 & 6\\
\hline
4 & 11 & 17 & 11 & $\infty$ & 12 & 11\\
\hline
5 & 5 & 5 & 5 & 5 & $\infty$ & 5\\
\hline
\end{tabular}


\item Вычтем минимальный элемент $d_{i}$ строки из каждого ее элемента:

\begin{tabular}{|c|c|c|c|c|c|}
\hline
& 1 & 2 & 3 & 4 & 5 \\
\hline
 1 & $\infty$ & 12 & 10 & 4 & 0 \\
\hline
2 & 0 & $\infty$ & 9 & 2 & 6 \\
\hline
3 & 6 & 12 & $\infty$ & 0 & 5 \\
\hline
4 & 0 & 6 & 0 & $\infty$ & 1 \\
\hline
5 & 0 & 0 & 0 & 0 & $\infty$ \\
\hline
\end{tabular}


\item Такую же операцию редукции проведем для столбцов. Найдем минимальный элемент $d_{j}$ для каждого столбца:

\begin{tabular}{|c|c|c|c|c|c|}
\hline
& 1 & 2 & 3 & 4 & 5 \\
\hline
 1 & $\infty$ & 12 & 10 & 4 & 0 \\
\hline
2 & 0 & $\infty$ & 9 & 2 & 6 \\
\hline
3 & 6 & 12 & $\infty$ & 0 & 5 \\
\hline
4 & 0 & 6 & 0 & $\infty$ & 1 \\
\hline
5 & 0 & 0 & 0 & 0 & $\infty$ \\
\hline
$d_{j}$ & 0 & 0 & 0 & 0 & 0 \\
\hline
\end{tabular}


\item Вычтем минимальный элемент $d_{j}$ столбца из каждого его элемента

\begin{tabular}{|c|c|c|c|c|c|}
\hline
& 1 & 2 & 3 & 4 & 5 \\
\hline
 1 & $\infty$ & 12 & 10 & 4 & 0 \\
\hline
2 & 0 & $\infty$ & 9 & 2 & 6 \\
\hline
3 & 6 & 12 & $\infty$ & 0 & 5 \\
\hline
4 & 0 & 6 & 0 & $\infty$ & 1 \\
\hline
5 & 0 & 0 & 0 & 0 & $\infty$ \\
\hline
\end{tabular}

\item После проделанных операций, в каждой строке и в каждом столбце гарантированно найдется элемент$a[i][j] = 0$. Для кажого такого элемента найдем коэффициент, который будет равен сумме минимального элемента в строке $i$ и столбце $j$. Затем удалим из матрицы строку и столбец, в которой будет находиться нулевой элемент с наибольшим коэффициентом, а обратному пути $a[j][i]$ (если он существует) присвоим бесконечно большое значение. Если же нулевых элементов с максимальным коэффициентом несколько, можно удалить любой. Координаты удаленного элемента $i, j$ будут являться ребром, составляющим кратчайший гамильтонов граф. Как видно из матрицы, приведенной ниже, искомый элемент имеет координаты (5, 2), следовательно, 5 строку и 2 столбец необходимо исключить.

\begin{tabular}{|c|c|c|c|c|c|}
\hline
& 1 & 2 & 3 & 4 & 5 \\
\hline
 1 & $\infty$ & 12 & 10 & 4 & 0(5) \\
\hline
2 & 0(2) & $\infty$ & 9 & 2 & 6 \\
\hline
3 & 6 & 12 & $\infty$ & 0(5) & 5 \\
\hline
4 & 0(0) & 6 & 0(0) & $\infty$ & 1 \\
\hline
5 & 0(0) & 0(6) & 0(0) & 0(0) & $\infty$ \\
\hline
\end{tabular}

\item Для получившейся матрицы повторим шаги 1 - 6, пока ее размерность не станет 1х1.

\begin{tabular}{|c|c|c|c|c|}
\hline
& 1 & 3 & 4 & 5 \\
\hline
 1 & $\infty$ & 10 & 4 & 0 \\
\hline
2 & 0 & 9 & 2 & $\infty$ \\
\hline
3 & 6 & $\infty$ & 0 & 5 \\
\hline
4 & 0 & 0 & $\infty$ & 1 \\
\hline
\end{tabular}

\item В итоге получим матрицу:

\begin{tabular}{|c|c|}
\hline
& 5 \\
\hline
 1 & 0 \\
\hline
\end{tabular}

Нетрудно заметить, что единственной незамкнутой парой вершин остались 1 и 5, следовательно, ребро (1, 5) составляет искомый гамильтонов граф
\end{enumerate}
В результате, получим, что искомый путь: 1 $\rightarrow$ 5 $\rightarrow$ 2 $\rightarrow$ 4 $\rightarrow$ 3, а его длина равна 43

\newpage

\section{Реализация}
Алгоритм был реализован на языке С++. Программа считывает из файла \texttt{<<input.txt>>} размеры матрицы смежности и ее содержимое, а ребра гамильтонова графа и его длину выводит в файл \texttt{<<output.txt>>}. \\
\begin{itemize}
\item Функция \texttt{reductionMatrix(vector<vector<double>> a)} производит редукцию матрицы а по столбцам и строкам.
\item Функция \texttt{deleteRC(vector<vector<double>> \&a, int i, int j)} удаляет из матрицы а строку с номером \texttt{i} и столбец с номером \texttt{j}. 
\item В функции \texttt{bbmethod(vector<vector<double>> \&a)} реализуется непосредственно сам алгоритм ветвей и границ. Функция производит редукцию матрицы а с помощью функции \texttt{reductionMatrix(vector<vector<double>> a)}, а затем находит нулевой элемент с наибольшим коэффициентом приведения. Затем этот элемент удаляется из матрицы с помощью функции \texttt{deleteRC(vector<vector<double>> \&a, int i, int j)} вместе со строкой и столбцом, в которых он находится. После от получившейся матрицы рекурсионно вызывается функция \texttt{bbmethod(vector<vector<double>> \&a)}, а критерием выхода является размерность матрицы. Как только она становится равна 1, функция выводит координаты оставшегося элемента и прекращает работу.
\end{itemize}

\newpage

\section{Код программы}

\begin{verbatim}
#define _CRT_SECURE_NO_WARNINGS
#include <iostream>
#include <vector>
#include <fstream>
#include <limits>
#include <locale>

using namespace std;

vector<vector<double>> init(66);
int min_way = 0;

vector <vector<double>> reductionMatrix(vector <vector<double>> a) {
    double red_const;
    for (int i = 1; i < a[0].size(); i++) {
        red_const = numeric_limits <double>::infinity();
        for (int j = 1; j < a[0].size(); j++)
            if (a[i][j] < red_const) red_const = a[i][j];
        for (int j = 1; j < a[0].size(); j++)
            a[i][j] -= red_const;
    }
    for (int i = 1; i < a[0].size(); i++) {
        red_const = numeric_limits <double>::infinity();
        for (int j = 1; j < a[0].size(); j++)
            if (a[j][i] < red_const) red_const = a[j][i];
        for (int j = 1; j < a[0].size(); j++)
            a[j][i] -= red_const;
    }
    return a;
}

void deleteRC(vector<vector<double>> &a, int i, int j) {
    int n = a[0].size();
    double b;
    for (int k = i; k < n - 1; k++)
        for (int q = 0; q < n; q++) {
            b = a[k][q];
            a[k][q] = a[k + 1][q];
            a[k + 1][q] = b;
        }
    a.resize(n - 1);
    for (int k = j; k < n - 1; k++)
        for (int q = 0; q < n - 1; q++) {
            b = a[q][k];
            a[q][k] = a[q][k + 1];
            a[q][k + 1] = b;
        }
    for (int i = 0; i < n - 1; i++)
        a[i].resize(n - 1);
}

void bbmethod(vector<vector<double>> &a) {
    if (a[0].size() > 2) {
        a = reductionMatrix(a);
        int maxi, maxj;
        double branch_edge_column, branch_edge_row, max_b_l = -1;
        for (int i = 1; i < a[0].size(); i++) {
            for (int j = 1; j < a[0].size(); j++) {
                if (a[i][j] == 0) {
                    branch_edge_column = numeric_limits <double>::infinity();
                    branch_edge_row = numeric_limits <double>::infinity();
                    for (int k = 1; k < a[0].size(); k++) {
                        if (a[i][k] < branch_edge_row && k != j) 
                            branch_edge_row = a[i][k];
                        if (a[k][j] < branch_edge_column && k != i) 
                            branch_edge_column = a[k][j];
                    }
                    if (branch_edge_column + branch_edge_row > max_b_l) {
                        max_b_l = branch_edge_column + branch_edge_row;
                        maxi = i; maxj = j;
                    }
                }
            }
        }
        a[maxi][maxj] = numeric_limits <double>::infinity();
        for (int i = 1; i < a[0].size(); i++)
            for (int j = 1; j < a[0].size(); j++)
                if (a[maxi][0] == a[0][j] && a[0][maxj] == a[i][0])
                    a[i][j] = numeric_limits <double>::infinity();
        printf("(%.0f, %.0f)\n", a[maxi][0], a[0][maxj]);
        min_way += init[a[maxi][0] - 1][a[0][maxj] - 1];
        deleteRC(a, maxi, maxj);
        bbmethod(a);
    } else {
        printf("(%.0f, %.0f)\n", a[1][0], a[0][1]);
        min_way += init[a[1][0] - 1][a[0][1] - 1];
    }
}

int main() {
    setlocale(LC_ALL, "Russian");
    ifstream fin("input.txt");
    ofstream fout("output.txt");
    freopen("input.txt", "r", stdin);
    freopen("output.txt", "w", stdout);
    vector<vector<double>> a(66);
    int n, x;
    cin >> n;
    for (int i = 0; i <= n; i++)
        a[0].push_back(i);
    for (int i = 1; i <= n; i++) {
        a[i].push_back(i);
        for (int j = 0; j < n; j++) {
            cin >> x;
            a[i].push_back(x);
            init[i - 1].push_back(x);
        }
    }
    for (int i = 1; i < a[0].size(); i++)
        a[i][i] = numeric_limits <double>::infinity();
    cout << "Ребра, составляющие кратчайший гамильтонов цикл:" << endl;
    bbmethod(a);
    cout << "Длина этого цикла: ";
    cout << min_way << endl;
    return 0;
}
\end{verbatim}

\section*{Итоги}
В результате данного исследования были изучены новые алгоритмы и методы комбинаторной оптимизации, а также улучшены навыки владения языком программирования С++ и системой подготоки документов LaTex.

\begin{thebibliography}{9}
\bibitem{sig02} Сигал И. Х., \emph{<<Введение в прикладное дискретное 
программирование: модели и вычислительные алгоритмы>>}, 2002
\bibitem{wiki} URL: \emph{<<Travelling salesman problem>>}, $https://en.wikipedia.org/wiki/Travelling\_salesman\_problem$
\bibitem{examplei} URL: \emph{<<Пример решения задачи коммивояжера>>}, $https://math.semestr.ru/kom/komm.php$
\bibitem{voron} Воронцов К. В. \emph{<<\LaTeX в примерах>>}, 2005
\end{thebibliography}

\end{document}